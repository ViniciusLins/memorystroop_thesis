This paper approaches the use of a diagnosis and therapeutic game for Attention Deficit Hyperactivity Disorder (ADHD) young patients in Salvador, state of Bahia, Brazil. In such interdisciplinary analysis, knowledges of computer sciences and psycho-pedagogy are combined to perform the main objective, which is to show correlation evidences of the use of the proposed game and improvements in attentions level measures of the disease, following Stroop test paradigms. The methods employed consist in statistical techniques in order to inquiry validation to the aimed hypothesis.



\begin{keywords}
Gamification, Mobile Application, Hypothesis Test, Educative and Therapeutic Software
\end{keywords}
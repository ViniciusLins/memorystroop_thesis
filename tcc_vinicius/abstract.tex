This undergraduate thesis approaches the construction of a diagnostic and therapeutic game for Attention Deficit Hyperactivity Disorder (ADHD) for young patients. ADHD presents a series of challenges, including many related problems for example dysfunctions on working memory due the correlations between attention problems and neural functions of memory. The psychological phenomenon of Stroop Effect exploits the attention faults when people are submitted to conflicting stimuli tasks and is employed in the proposed application because Stroop Effect is a common mean to diagnose those patients. Combining a memory game for exercising working memory health with Stroop Effect tests for exercising attention tasks, the proposed application in this undergraduate thesis, called \textit{Memory Stroop}, presents itself as a project of gamification. A gamified system is a application that implements a serious process of human daily life in a pleasant game-like way. This gamified alternative has been used in many fields of knowledge, market and health services in order to turn fairer very rigorous tasks. The theoretical basis and principles needed to implement that application is found mainly in psychology and computer sciences Literature. In such interdisciplinary analysis, knowledges of Computer Sciences and Psycho-pedagogy are combined to perform the main objective, which is to show correlation evidences of the use of that gamification software and possible improvements in attentions level measures of the disease, following Stroop Effect paradigms. At the end of the research, an experimental case study has been conducted to verify the viability of \textit{Memory Stroop} among computer scientists.



\begin{keywords}
Gamification, Mobile Application, Educative and Therapeutic Software
\end{keywords}
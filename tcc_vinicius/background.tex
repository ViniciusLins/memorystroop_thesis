\section{Background}
\label{background}
For a complete understanding of this research, as well as the ambitions related to the areas involved, it is necessary to understand the main issues: gamification, social computing and civil construction, in addition to concepts related to well-being at work and the involvement of employees. The themes are presented in different parts, and the most relevant aspects are discussed.

\subsection{Social Computing}
Social computing is a general term used in the field of Computer Science to address the receipt, processing and dissemination of information about social behavior through computer systems~\cite{Masunaga}. Social computing has become an important concept in business. In ~\cite{Masunaga}, in the weak sense of the term, Social Computing is the support of any type of social behavior through computer systems. It is based on creating social contexts with software and technology. Blogs, email, instant messaging, social networking services, wikis, social bookmarking and other software with which people can interact socially represent the idea of social computing. 

%(II) In the strongest sense of the term, social computing has to do with the computer supports the work done by groups of people, an idea that was popularized in the book of James Surowiecki, ``The Wisdom of Crowds''. Hence the synergy of social computing to the present research.
 
According to~\cite{IPTS}, Social Computing is already viewed as an important social phenomenon in terms of its scope, use of time and activities. In the context of this work, the principles of social computing are applied to engage workers at a construction site with a serious web based game. Thus, we hope to achieve positive results beyond the game itself, e.g., socialization and joy. We believe that this collective engagement will push workers to be  more motivated and effective at work. 

%In this research, the acceptance by the community of the game will be a key success factor. The interaction with the participants of the game and the team work during the case study will be of great importance. The use of the concept of social computing in performing this work will help in understanding how gamification can accelerate social engagement. The use of the technologies proposed in this study will facilitate the understanding of the activities that should be undertaken by employees and provide daily monitoring of their performance in their work in a playful way through Gamification.

%%%%%%%%%%%%%%%%%%%%%%%%%%%%%%%%%%%%%Corrigido

\subsection{Gamification}
Gamification has emerged as a new trend in Information Systems and is defined as the use of game design elements in non-gaming contexts~\cite{Deterding}.
According to~\cite{Smith}, play is the preferred way of learning and challenging the human brain because the thought of play is the opposite of seriousness, and most people do not realize that most relationships are governed through board games, political games, law, financial games, love games, advertising, etc.

The spirit of play is fundamental to human life and society, inspiring the visual, musical, and verbal arts and other elements that are valued by the culture. Games have the ability to develop a state of balance, creativity and objectivity. In addition, \cite{Liu} suggests that the ultimate goal of gamification is to incentive a non-game system user to have so-called game-like behavior: focusing on the task at hand, multitasking under pressure, working overtime without a discontented attitude, not accepting failure, etc.

In~\cite{Zich}, the ``gamified'' professional develops characteristics such as leadership, cooperation, competitiveness, the ability to work toward targets and to make quick decisions; human beings are competitive by nature and perform their duties better if they are approached in a playful and fun way or even if there is some social compensation involved, as a motivational aspect to overcome their own limitations.

\subsection{Civil Construction}
In recent years, the construction industry has shown significant interest in new process improvement principles and practices that use visual management, which was also integrated into the Production Planning and Control at construction sites. This need has led managers to seek out new approaches to improve the involvement of their employees in complying with the weekly plan.% The competition between construction companies and labor shortages  the need to train and motivate those involved to meet the market demand for high productivity and quality.

Lean Construction was named by~\cite{Koskela} as a new management philosophy for production; it originated from the Toyota Production System - STP and was adapted for construction. This system of production increased competitiveness by identifying and eliminating waste in job site services. Losses are not limited to defective products in the mass production system, but also include loss of resources, manpower and equipment in activities that do not add value to the final product. To reduce losses,~\cite{Koskela} proposes the principle of transparency, which is defined as a visually oriented production control principle that makes production processes more observable for those involved in the project~\cite{Womack}. To promote transparency, this research adopts visual management, this technique uses visual aids to improve communication processes and promote continuous improvement~\cite{Cicmil}.

The development of a computer system based on techniques of gamification and associated with the construction environment is directly linked to the Lean philosophy; thus, the software will provide better communication between those involved in the work. Ultimately, this will actually achieve a significant gain in productivity, thereby reducing losses due to process.

%With social computing and gamification, it is possible to develop a system of support for visual management based on the use of the gamification techniques mentioned above. The goal is to obtain productivity gains in relation to the management methods currently used.

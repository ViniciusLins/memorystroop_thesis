\section{Introduction}
\label{introduction}

Gamification is the use of game elements, such as mechanics, in non-game contexts for the purpose of engaging users to perform determined actions and making them have more fun in the process ~\cite{2212883,Huotari,Zich}. The aim of this technique is to make boring or repetitive activities more pleasant. This is usuallyintended to create gameful experiences, motivate desired user behaviors, and generally, increase joy of use, ~\cite{2212883,Huotari} . Recent research conducted by  ~\cite{Gartner}, shows that gamification has become an important part of the business strategy of many companies, for the purpose of engaging employees and retaining customers.

The insertion of game mechanics into a corporative environment makes the learning process easier and more fun~\cite {Medina}. Games can naturally motivate by entertaining, engaging users with rewards and stimuli, and training by involvement and learning. The ``gamified'' professional develops characteristics like leadership, cooperation, competitiveness and the ability to work with goals and make quick decisions~\cite{Zich}. Humans are naturally competitive and they are more capable of accomplishing tasks when they are fun or a social reward is involved~\cite{Medina}. For this reason companies have applied gamification activities to engage thier employees to carry out regular jobs with more fun.

The field of Civil Engineering is constantly expanding, and the technology related to this area of study is also growing. Investment in innovation at the construction site is now necessary to improve communication among the many factors on the production chain; it helps improving control over the labor supply, materials and other resources. Production planning and control is essential, as they have a large impact on productivity and ensure that deadlines will be respected~\cite{formoso}. Additionally, most services in civil construction do not use a production line to improve productivity, unlike other industries in which many repetitive tasks are automatized. Therefore, Gamification encourages work engagement and facilitates the management of the construction site. 

% Most services in civil construction are handmade, unlike other areas of industry where a production line is used, it make these tasks repetitive and no motivational to the worker, and hard to be managed. There are already case studies of ludification of activities in some sectors of industry to improve the employee's performance \cite {Medina}, however, it have not been applied in the construction site. Therefore, gamification is an option to make the worker more engaged to his tasks and make the management easier with the visual results of each collaborator. 

Based on these concepts, we propose a Gamification system to visually assist, managers and workers, with their activities at the construction site. We intend to evaluate how this Gamification system encourages civil construction workers to achieve weekly goals established with short-term programming and how managers can benefit from workers who are more motivated and focused, as well as the difficulties in the implementation of the system and how the system impacts the productivity of workers.

This work is structured as follows: In Section \ref{background}, we describe the concepts that support this proposal including gamification, social computing and civil construction; in Section \ref{relatedwork} we present related works; Section \ref{gamifiedsystem} depicts the gamification system and how the gamification rules are set, and in Section \ref{experimentalEvaluation} we present an evaluation of the system at the construction site and discuss points of improvement; and finally, in Section \ref{conclusion}, we conclude and present ideas for future work.

%Marivaldo
%- Introduzir os conceitos essenciais do paper: computção social, gameficação e contextualizar esses conceitos na construção civil
%- Falar sobre problemas de contrução civil
%Motivar o paper mostrando:
%- que os colaboradores não tem a visao de toda a obra. O sistema pode ajudar nisso.
%- Outro ponto  da questão da competitividade saudável entre os colaboradores.
%Por isso... necessário criar uma um sistema de informação gamificado de apoio a planejamento e execução de obras em um canteiro de obras.

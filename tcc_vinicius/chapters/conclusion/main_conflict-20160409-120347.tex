\chapter{Conclusion}
\label{ch:conclusion}

A partir dos resultados levantados com os question\'{a}rios aplicados, conclui-se que o jogo Cores Beta alcan\c{c}ou o seu objetivo de ser um jogo, para tablets e smartphones Android, que possa ser usado no tratamento de crian\c{c}as com TDAH, pois os especialistas de psicologia foram unanimes em indicar que acreditam nesse potencial do Cores Beta. Tamb\'em pode-se identificar que o jogo apresenta uma boa intera\c{c}\~{a}o com o seu jogador, tomando-se como base as respostas obtidas pelos especialistas em Computa\c{c}\~{a}o.

Por fim, pode-se identificar que o jogo \'e divertido para crian\c{c}as, caracteristica muito importante, j\'{a} que pretende-se melhorar as capacidades cognitivas de crian\c{c}as com TDAH. Tamb\'em, que as regras do jogo podem ser entendidas pelas crian\c{c}as e que sua dificuldade parece ser adequada. Contudo, muitas s\~{a}o as poss\'{i}veis mudan\c{c}as que o jogo pode sofrer para ser mais adequado ao seu objetivo, existindo uma grande \'{a}rea de atua\c{c}\~{a}o \`{a} ser explorada.

\subsection{Coming studies}

In order to increment the potential of the gamification proposed on Memory Stroop and offer it as a future mean of diagnosis and therapy for attention related it is necessary to sum up some core aspects that turns the application usable and medically effective. The following list include newer features, research aspects and implementation resources to be achieved:

\begin{itemize}
\item Monitor cerebral activities during game sections in further studies using Memory Stroop, by which it will be possible to trace neurological evidences that could corroborate the gaming therapy proposed here;
\item Extend the practical research to a really quantitative research with human beings, having 50 or 100 young people by double-blind randomized sampling and examine a high quantity of variables with machine learning algorithms in a Knowledge Discover in Database approach;
\item Perform a improvement in the game graphs, sound effects, engine and other usability features;
\item Improve application portability to other mobile operating systems than android or desktop platforms;
\item Contribute to the formation of a group of study on ADHD gamification treatment including researchers and professionals of many areas, namely, Computer Sciences, Psychology, Medicine and Pedagogy.


\end{itemize}

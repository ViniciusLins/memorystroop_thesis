\chapter{Conclusion}
\label{ch:conclusion}

ADHD challenges in contemporary society beyond common diagnosis and treatments include a fertile soil for gamification like \textit{Memory Stroop}. It is designed as an Android mobile game for diagnostic and therapeutic processes that combines tasks for detection and correction of working memory defaults common for ADHD children. It was possible see has its functional and non-functional requirements extends and improves a previous application developed by \cite{Villa} and validated by pyschologists, who let important considerations for the present project. An practical evaluation was performed with computer scientists, and show the positive aspects and negative aspects of implementation.

Thus, the present gamification project has laid foundations for other studies that want to combine gaming and serious purposes.  As other fields, for example, civil engineering business and human resources division of certain company, gamification in ADHD diagnosis and treatment of memory ADHD malfunctions is a good issue for Psychology/Psychiatric, Computer Sciences and Pedagogics. Gamification is a promising field of Information Technology \& Systems, converting so-called unpleasant tasks into more agreeable activities presented in fashion of video game characteristics. 

The Stroop Effect consists in a concentration fault to target correct objective such as saying a color written name instead to saying another different color name ink that lies on background. This conflicting stimuli is precisely feel for ADHD patients, because they present a relative lack in their memory functions.

After basing the theoretical fundamentals on correlate works, that stressed the importance to offer alternatives less invasive to the common ways of treatment, this graduate thesis has presented how gamification can help ADHD children and teenagers with simple but considerable initiatives.

\section{Future works}

In order to increment the potential of the gamification proposed on Memory Stroop and offer it as a future mean of diagnosis and therapy for attention related it is necessary to sum up some core aspects that turns the application usable and medically effective. The following list include newer features, research aspects and implementation resources to be achieved:

\begin{itemize}
	\item Monitor cerebral activities during game sections in further studies using \textit{Memory Stroop}, by which it will be possible to trace neurological evidences that could corroborate the gaming therapy proposed here;
	\item Develop the practical research as a really quantitative research with human beings, having 50 or 100 young people by double-blind randomized sampling and examine a high quantity of variables with machine learning algorithms in a Knowledge Discover in Database approach;
	\item Perform a improvement in the game graphs, sound effects, engine and other usability features;
	\item Improve application portability to other mobile operating systems than android or desktop platforms;
	\item Contribute to the formation of a group of study on ADHD gamification treatment including researchers and professionals of many areas, namely, Computer Sciences, Psychology, Medicine and Pedagogy.
	
	
\end{itemize}
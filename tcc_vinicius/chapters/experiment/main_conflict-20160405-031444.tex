\chapter{Experiment}
\label{ch:caseStudy}

In this chapter, it will presented the practical approach object of study. In the first section, the subject is the methodology, explaining about the main aspects of our fieldwork and the nature of the practical research to be applied with children. The conduction of this research is only presented in the next chapter followed some institutional constraints that are also presented in it.  After presenting theoretical basis as we did in previous chapter, it is time to consolidate for validating it with its targeted final users, in this case, ADHD children. As we underlined before, a gamiefied application have to deal, to balance funny features with serious aims in order to turn practical but awful tasks more pleasant. This is the main principle of this thesis, that can be shown in every section of the present chapter.


\section{Methodology}

In this section we presents the methodological definitions and theirs justifications for this research. The object of the present study comprises the development of a tool for detecting and treating the ADHD. This tool follows the notion of gamification, as discussed in previous chapter. Therefore, the main outcome of the research is to offer a game-like mobile implementation thats contributes to children's mental health. In a way of explain its methodological features, the first subsection treats upon the employed means and materials for developing the game-like application and the second one discuss how the precise method of case study is applied along this research.

\subsection{A case study approach and its implications}

Following a cyclical view of process as stated by software Engineering \citep{pressman}, our application all over the implementation sprints has been frequently submitted to its potential user as way to improve the gamified applications. So that, the method choose to that development is the case study.

As some authors stated \citep{Lauri-2011}, the case study is a powerful tool of analysis of software production. It could, but the same authors employ the qualitative approach, because it promotes fast answers without the necessity of tabulation or other statistical that may cost high computational processing, monetary investment and time. 

In this particular aspect of the methodological choice for a qualitative case study approach we follow them, for the reduced resources and the lack of extensive institutional infrastructure of our University Psychological services' for ADHD children, desired in a extensive quantitative analysis.

Since a case study requires the net definition of its tasks \citep{Lauri-2011}, we present them in sequence. To starts, we perform the research with four children diagnosed with ADHD and another four not-diagnosed at the Psychological Clinics of Federal University of Bahia. This research is applied in April 2016. 

There is no randomization or other advanced techniques in composing the sample: it is a common convenience sample. Collected data were made by observation and by automated data monitors inserted in application. These are generated as a report legible both by Psychologists and System Analysts.

Moreover, as the research involves human beings, particularly, children, first of all it has been necessary to ask permission of their parents, of the Institution (by a predefined form) and work together with psychological personnel of the Clinics. All documentation and report models may be seen on the Appendix of this work.

The identities (names and other personal information) of children were protected by pseudonyms and identification numbers (IDs)  in order to avoid over-exposition. For each automated monitored playing, the respective ID of the user is stored in the header of common data. In this way, it is possible to track each monitored playing outcomes with the child without employing its name or other specific data.

The data collected from children are both qualitative (their behavior during playings, reactions, doubts, suggests and general impressions about the gaming application after that) and technical (time of playing, level achieved etc). These massive information is analyzed and general considerations made upon them.

The collected data offer three utilities: 
\begin{enumerate}
\item Validates the implementation with target-users;
\item Verifying differences in use of ADHD and non-ADHD children;
\item Collect information for improve the application in newer versions.
\end{enumerate}
	
The results of each one of these aimed features are presented in the next chapter. Evidences should support consistent validations or readjustments in our application, that is another feature for the next chapter. By now, it is necessary to understand the organization of software features. 

\section{Results}

\section{Discussion}

\section{Summary}

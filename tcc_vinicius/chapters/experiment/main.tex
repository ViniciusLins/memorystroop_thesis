\chapter{Experiment}
\label{ch:experiment}

After presenting theoretical basis and the application's development in previous chapters, it is time to evaluate \textit{Memory Stroop} with specialists. A gamified application have to combine pleasant tasks and serious aims in order to turn practical but awful tasks more pleasant. This is the main topic of the experiment, that can be shown in every section of the present chapter. First it is presented the applied format of case study, after that the questionnaire, following it the results and finally discussion.

\section{Nielsen's heuristics and usability evaluation}

\texttt{%\section{A Case Study Approach and Implications}
%
%The case study proposed here is based on a survey of usability analysis features to check and improve the gamified applications.Therefore the method chosen to that development is the case study. Some authors emphasize usability is a diffuse parameter of software quality that treats not only the comprehensiveness of the user interface, as well as the learnability, efficiency, reliability and satisfaction \citep{Folmer2005}. Generally, final user does not know how to measure each one of those  components. That is why usability analysis is performed with specialists, as in the present study. Doolin states that the case study is a powerful tool of analysis of software production \citep{Lauri-2011}. It could, but the same authors employ the qualitative approach, because it promotes fast answers without the necessity of tabulation or other statistical that may cost high computational processing, monetary investment and time. 
%
%\section{Description of the Present Case Study}
%
%The quantitative and qualitative research does not involves data of children, only survey data about usability analysis measures. No personal information was collected and, because of this reason, there is no necessity of submission to University's Ethics Council. The questionnaire contains items about viability of gamification, simplicity/complexity of tasks and so on.
%	
%It has been performed with five computer science specialists, in May 2016. The general profile of the answerers involves people of both sex, at age of 25 to 32 years-old and with a professional background on mobile programming. There is no randomization or other advanced techniques in composing the sample: it is a common convenience sample. Data collection is made by a written survey shown  above. All sessions last 20 minutes, are monitored physically by the author, happens on the students' computer sciences laboratory and consist in three activities as it follows: the author's briefly explanation, the answerer's playing and finally the answerer's respond to the questionnaire. 
%
%The identities (names and other personal information) of specialists are	 protected by identification numbers (IDs) on questionnaires in order to avoid over-exposition. For each automated monitored playing, the respective ID of the user is stored in the header of common data. In this way, it is possible to track each monitored playing outcomes with the child without employing its name or other specific data. The collected data offer three utilities: 
%\begin{enumerate}
%\item Validate the implementation with experiment evaluators;
%\item Verify the gamification effectiveness;
%\item Collect information for improve the application in newer versions.
%\end{enumerate} 
%
%\subsection{Questionnaire and Comparative Approach}
%
%All questions are derived from Villa's previous research for its own application called `Colors' \citep{Villa}, and its results are compared with present experiments' results to compare usability patterns between the two application. Additionally, Villa's experiment involved more people, 10 specialists and the same number of two other groups (psychologists and children) that are not consulted in this research by reason of lack of enough time to perform evaluation with them. The comparison is made involving the group of computer scientists of each research. The proposed questionnaire has 11 questions. The same questions of both experiments are shown bellow.
%
%\subsection{Questionnaire Responded by Computer Sciences Specialists}
%
%\begin{enumerate}
%	\item The progressiveness of difficult is adequate.
%	\subitem () totally disagree
%	\subitem () disagree
%	\subitem () no position
%	\subitem () agree
%	\subitem () totally agree
%
%	\item  It is possible to understand the game rules.
%	\subitem () totally disagree
%	\subitem () disagree
%	\subitem () no position
%	\subitem () agree
%	\subitem () totally agree
%	
%	\item The game is not functional, it is slow.
%	\subitem () totally disagree
%	\subitem () disagree
%	\subitem () no position
%	\subitem () agree
%	\subitem () totally agree
%
%	\item The proposed gamification is satisfactory.
%	\subitem () totally disagree
%	\subitem () disagree
%	\subitem () no position
%	\subitem () agree
%	\subitem () totally agree
% 
%	\item The disposition of objects on screen is confuse.
%	\subitem () totally disagree
%	\subitem () disagree
%	\subitem () no position
%	\subitem () agree
%	\subitem () totally agree
%	
%	\item The game informs the player situation well.
%	\subitem () totally disagree
%	\subitem () disagree
%	\subitem () no position
%	\subitem () agree
%	\subitem () totally agree
%	
%	\item The screen sequence is confuse.
%		\subitem () totally disagree
%		\subitem () disagree
%		\subitem () no position
%		\subitem () agree
%		\subitem () totally agree
%
%	\item Have you needed some help to understand the game?
%	\subitem () Yes
%	\subitem () No
%	\item Does the game present any error?
%	\subitem () Yes
%	\subitem () No	
%	
%	\item Which errors do you have found? (Subjective)
%	
%	\item How to improve the game in order to collaborate with ADHD children inhibitory control mobility? 
%	(Subjective)
%	
%\end{enumerate}
%
%\section{Results}
%In this section it is presented the questions, results and their percentages on charts. From the first to the seventh, there are five options: `totally disagree', `disagree', `no position', `agree' and `totally agree'. On charts, each of these options are respectively indicated by numbers 1 to 5 on the horizontal axis. Questions 8 and 9 have only `yes' or `no' represented naturally as 1 or 2 on its charts. Questions 10 (optional) and 11 are subjective. Results  of two applications are briefly compared.
%
%\subsubsection{Question One}
%
%The first question (``The progressiveness of difficult is adequate.'') aims to check if the progression of difficult is playable. On \textit{Memory Stroop} results, the majority of answers express that it is really adequate. As the report below shows, it can be seen the absence of totally disagreement. The results are extensively close to Villa's Colors results \citep{Villa}, although there is a higher score for \textit{Colors} evaluation.
%
%\noindent
%\begin{minipage}[t]{.5\textwidth}
%\raggedright
%\begin{tikzpicture}
%\begin{axis}[%
%scale only axis,
%width=2in,
%height=2in,
%xmin=0, xmax=5,
%ymin=0, ymax=100,
%axis on top]
%\addplot[
%  ybar,
%  bar width=0.102874in, 
%  fill=gray,
%  draw=black] 
%  plot coordinates{ 
%    (1,0) (2,20)     (3,0) (4,60)     (5,20)
%  };
%
%\end{axis}
%\end{tikzpicture}
%\textit{Memory Stroop}
%\end{minipage}% <---------------- Note the use of "%"
%\begin{minipage}[t]{.5\textwidth}
%\raggedleft
%\begin{tikzpicture}
%\begin{axis}[%
%scale only axis,
%width=2in,
%height=2in,
%xmin=0, xmax=5,
%ymin=0, ymax=100,
%axis on top]
%\addplot[
%  ybar,
%  bar width=0.102874in, 
%  fill=gray,
%  draw=black] 
%  plot coordinates{ 
%    (1,0) (2,0)     (3,0) (4,20)     (5,80)
%  };
%
%\end{axis}
%\end{tikzpicture}
%\textit{Colors}
%\end{minipage}
%
%
%\subsubsection{Question Two}
%
%The second question (``It is possible to understand the game rules.'') is related to understanding of game rules majority. By the same results of previous question the answers express that learneability of game rules are totally adequate. There is higher scores for \textit{Colors} upon  \textit{Memory Stroop} (70 \% for option `totally agree' upon 60 \%).
%
%\noindent
%\begin{minipage}[t]{.5\textwidth}
%\raggedright
%\begin{tikzpicture}
%\begin{axis}[%
%scale only axis,
%width=2in,
%height=2in,
%xmin=0, xmax=5,
%ymin=0, ymax=100,
%axis on top]
%\addplot[
%  ybar,
%  bar width=0.102874in, 
%  fill=gray,
%  draw=black] 
%  plot coordinates{ 
%    (1,0) (2,0)     (3,0) (4,40)     (5,60)
%  };
%
%\end{axis}
%\end{tikzpicture}
%\textit{Memory Stroop}
%\end{minipage}% <---------------- Note the use of "%"
%\begin{minipage}[t]{.5\textwidth}
%\raggedleft
%\begin{tikzpicture}
%\begin{axis}[%
%scale only axis,
%width=2in,
%height=2in,
%xmin=0, xmax=5,
%ymin=0, ymax=100,
%axis on top]
%\addplot[
%  ybar,
%  bar width=0.102874in, 
%  fill=gray,
%  draw=black] 
%  plot coordinates{ 
%    (1,0) (2,0)     (3,0) (4,30)     (5,70)
%  };
%
%\end{axis}
%\end{tikzpicture}
%\textit{Colors}
%\end{minipage}
%
%\subsubsection{Question Three}
%
%In the third topic of the survey (``The game is not functional, it is slow.''), the topic talks about the efficiency/slowly of the game. The results of the present application are real next to Villa's counterparts in this specific measure only in position, but Villa's best score, 70\% at `totally disagree' is only 10\% higher than \textit{Memory Stroop} equivalent.
%
%\noindent
%\begin{minipage}[t]{.5\textwidth}
%\raggedright
%\begin{tikzpicture}
%\begin{axis}[%
%scale only axis,
%width=2in,
%height=2in,
%xmin=0, xmax=5,
%ymin=0, ymax=100,
%axis on top]
%\addplot[
%  ybar,
%  bar width=0.102874in, 
%  fill=gray,
%  draw=black] 
%  plot coordinates{ 
%    (1,60) (2,20)     (3,20) (4,0)     (5,0)
%  };
%
%\end{axis}
%\end{tikzpicture}
%\textit{Memory Stroop}
%\end{minipage}% <---------------- Note the use of "%"
%\begin{minipage}[t]{.5\textwidth}
%\raggedleft
%\begin{tikzpicture}
%\begin{axis}[%
%scale only axis,
%width=2in,
%height=2in,
%xmin=0, xmax=5,
%ymin=0, ymax=100,
%axis on top]
%\addplot[
%  ybar,
%  bar width=0.102874in, 
%  fill=gray,
%  draw=black] 
%  plot coordinates{ 
%    (1,70) (2,30)     (3,0) (4,0)     (5,0)
%  };
%
%\end{axis}
%\end{tikzpicture}
%\textit{Colors}
%\end{minipage}
%
%\subsubsection{Question Four}
%
%Afterwards the question 4 (``The proposed gamification is satisfactory.'') is about the quality of gamification. In both applications the same majority agree to the gamification proposals but \textit{Memory Stroop} has earned a higher score at `totally agree' option (60\%) than \textit{Colors} (30\%).
%
%% 4
%
%\noindent
%\begin{minipage}[t]{.5\textwidth}
%\raggedright
%\begin{tikzpicture}
%\begin{axis}[%
%scale only axis,
%width=2in,
%height=2in,
%xmin=0, xmax=5,
%ymin=0, ymax=100,
%axis on top]
%\addplot[
%  ybar,
%  bar width=0.102874in, 
%  fill=gray,
%  draw=black] 
%  plot coordinates{ 
%    (1,0) (2,0)     (3,20) (4,20)     (5,60)
%  };
%\end{axis}
%\end{tikzpicture}
%\textit{Memory Stroop}
%\end{minipage}% <---------------- Note the use of "%"
%\begin{minipage}[t]{.5\textwidth}
%\raggedleft
%\begin{tikzpicture}
%\begin{axis}[%
%scale only axis,
%width=2in,
%height=2in,
%xmin=0, xmax=5,
%ymin=0, ymax=100,
%axis on top]
%\addplot[
%  ybar,
%  bar width=0.102874in, 
%  fill=gray,
%  draw=black] 
%  plot coordinates{ 
%    (1,0) (2,0)     (3,10) (4,60)     (5,30)
%  };
%
%\end{axis}
%\end{tikzpicture}
%\textit{Colors}
%\end{minipage}
%
%
%\subsubsection{Question Five}
%
%Question five (``The disposition of objects on screen is confuse.'') is related to the screen changes presented and \textit{Memory Stroop} has a higher score on option `totally disagree' in comparison to Colors (60\% upon 33\%).
%
%\noindent
%\begin{minipage}[t]{.5\textwidth}
%\raggedright
%\begin{tikzpicture}
%\begin{axis}[%
%scale only axis,
%width=2in,
%height=2in,
%xmin=0, xmax=5,
%ymin=0, ymax=100,
%axis on top]
%\addplot[
%  ybar,
%  bar width=0.102874in, 
%  fill=gray,
%  draw=black] 
%  plot coordinates{ 
%    (1,60) (2,40)     (3,20) (4,0)     (5,0)
%  };
%
%\end{axis}
%\end{tikzpicture}
%\textit{Memory Stroop}
%\end{minipage}% <---------------- Note the use of "%"
%\begin{minipage}[t]{.5\textwidth}
%\raggedleft
%\begin{tikzpicture}
%\begin{axis}[%
%scale only axis,
%width=2in,
%height=2in,
%xmin=0, xmax=5,
%ymin=0, ymax=100,
%axis on top]
%\addplot[
%  ybar,
%  bar width=0.102874in, 
%  fill=gray,
%  draw=black] 
%  plot coordinates{ 
%    (1,33) (2,55)     (3,11) (4,0)     (5,0)
%  };
%\end{axis}
%\end{tikzpicture}
%\textit{Colors}
%\end{minipage}
%
%
%\subsubsection{Question Six}
%
%Question six (``The game informs the player situation well.'') is about player situation and shows a higher score in \textit{Memory Stroop} (40 \% of `totally agree' vis-a-vis to 30 \% in \textit{Colors}). 
%
%\noindent
%\begin{minipage}[t]{.5\textwidth}
%\raggedright
%\begin{tikzpicture}
%\begin{axis}[%
%scale only axis,
%width=2in,
%height=2in,
%xmin=0, xmax=5,
%ymin=0, ymax=100,
%axis on top]
%\addplot[
%  ybar,
%  bar width=0.102874in, 
%  fill=gray,
%  draw=black] 
%  plot coordinates{ 
%    (1,0) (2,10)     (3,40) (4,20)     (5,40)
%  };
%
%\end{axis}
%\end{tikzpicture}
%\textit{Memory Stroop}
%\end{minipage}% <---------------- Note the use of "%"
%\begin{minipage}[t]{.5\textwidth}
%\raggedleft
%\begin{tikzpicture}
%\begin{axis}[%
%scale only axis,
%width=2in,
%height=2in,
%xmin=0, xmax=5,
%ymin=0, ymax=100,
%axis on top]
%\addplot[
%  ybar,
%  bar width=0.102874in, 
%  fill=gray,
%  draw=black] 
%  plot coordinates{ 
%    (1,0) (2,10)     (3,10) (4,50)     (5,30)
%  };
%
%\end{axis}
%
%\end{tikzpicture}
%\textit{Colors}
%\end{minipage}
%
%
%\subsubsection{Question Seven}
%
%Next, the seventh question (``The screen sequence is confuse.'') asks if exists problems on screen objects' positioning (one more time, a `negative' question). The result scores are higher in `totally disagree' on Memory Stroop game (40\%) in comparison to Villa's Colors (10\%).
%
%\noindent
%\begin{minipage}[t]{.5\textwidth}
%\raggedright
%\begin{tikzpicture}
%\begin{axis}[%
%scale only axis,
%width=2in,
%height=2in,
%xmin=0, xmax=5,
%ymin=0, ymax=100,
%axis on top]
%\addplot[
%  ybar,
%  bar width=0.102874in, 
%  fill=gray,
%  draw=black] 
%  plot coordinates{ 
%    (1,60) (2,20)     (3,20) (4,0)     (5,0)
%  };
%
%\end{axis}
%\end{tikzpicture}
%\textit{Memory Stroop}
%\end{minipage}% <---------------- Note the use of "%"
%\begin{minipage}[t]{.5\textwidth}
%\raggedleft
%\begin{tikzpicture}
%\begin{axis}[%
%scale only axis,
%width=2in,
%height=2in,
%xmin=0, xmax=5,
%ymin=0, ymax=100,
%axis on top]
%\addplot[
%  ybar,
%  bar width=0.102874in, 
%  fill=gray,
%  draw=black] 
%  plot coordinates{ 
%    (1,10) (2,60)     (3,30) (4,11)     (5,0)
%  };
%
%\end{axis}
%\end{tikzpicture}
%\textit{Colors}
%\end{minipage}
%
%
%\subsubsection{Questions Eight and Nine -- Bivalent Questions}
%
%
%As pointed before, questions 8 and 9 are bivalent. The first of them (``Have you needed some help to understand the game?'') is about, one more time, about learneability, but specifically about the necessity of help in question eight. Both games presented the same total result: 100 \% necessity of help.
%
%\noindent
%\begin{minipage}[t]{.5\textwidth}
%\raggedright
%\begin{tikzpicture}
%\begin{axis}[%
%scale only axis,
%width=2in,
%height=2in,
%xmin=0, xmax=5,
%ymin=0, ymax=100,
%axis on top]
%\addplot[
%  ybar,
%  bar width=0.102874in, 
%  fill=gray,
%  draw=black] 
%  plot coordinates{ 
%    (1,0) (2,100)
%  };
%
%\end{axis}
%\end{tikzpicture}
%\textit{Memory Stroop}
%\end{minipage}% <---------------- Note the use of "%"
%\begin{minipage}[t]{.5\textwidth}
%\raggedleft
%\begin{tikzpicture}
%\begin{axis}[%
%scale only axis,
%width=2in,
%height=2in,
%xmin=0, xmax=5,
%ymin=0, ymax=100,
%axis on top]
%\addplot[
%  ybar,
%  bar width=0.102874in, 
%  fill=gray,
%  draw=black] 
%  plot coordinates{ 
%    (1,0) (2,100)
%  };
%
%\end{axis}
%\end{tikzpicture}
%\textit{Colors}
%\end{minipage}
%
%
%In the question 9 (``Does the game present any error?'') asks if any error was detected the applation. The results of \textit{Memory Stroop} are 100\% of no errors on both applications meanwhile 20\% of specialists who have been consulted by Villa to evaluate Colors.
%
%\noindent
%\begin{minipage}[t]{.5\textwidth}
%\raggedright
%\begin{tikzpicture}
%\begin{axis}[%
%scale only axis,
%width=2in,
%height=2in,
%xmin=0, xmax=5,
%ymin=0, ymax=100,
%axis on top]
%\addplot[
%  ybar,
%  bar width=0.102874in, 
%  fill=gray,
%  draw=black] 
%  plot coordinates{ 
%    (1,0) (2,100)
%  };
%
%\end{axis}
%\end{tikzpicture}
%\textit{Memory Stroop}
%\end{minipage}% <---------------- Note the use of "%"
%\begin{minipage}[t]{.5\textwidth}
%\raggedleft
%\begin{tikzpicture}
%\begin{axis}[%
%scale only axis,
%width=2in,
%height=2in,
%xmin=0, xmax=5,
%ymin=0, ymax=100,
%axis on top]
%\addplot[
%  ybar,
%  bar width=0.102874in, 
%  fill=gray,
%  draw=black] 
%  plot coordinates{ 
%    (1,0) (2,80)
%  };
%
%\end{axis}
%\end{tikzpicture}
%\textit{Colors}
%\end{minipage}
%
%\subsubsection{Question Ten and Eleven}
%
%As question 10 is optional and no answer has been registered,  there is nothing to add about this question. The last and question number 11 (``How to improve the game in order to collaborate with ADHD children inhibitory control mobility?'') provided some answers that could be summarized in the necessity to improvement of game screens design, as well as in Villa's research.
%
%\section{Discussion}
%In questions 1, 2 and 3 previous Villa's application has presented higher scores on the most relevant option, meanwhile in the questions 4, 5, 6, 7 and 8  \textit{Memory Stroop} scores' has reveled best results. This shows that \textit{Colors} is a best tool in some aspects, like in progressiveness of difficult, the understanding of rule and about the functional character of the game. Other questions show best results for \textit{Memory Stroop} like screen sequence and positioning of objects on screen. But the most relevant advantage of \textit{Memory Stroop} upon \textit{Colors}  is the greater satisfaction about gamification proposal in the former. This suggests that it is positive that \textit{Memory Stroop} implements an evolution of research to a game without negative stimuli and centered on working memory with a simple memory game combined to Stroop test tasks.
}
\section{Summary}

In order to verify the proposed application's viability and core usability facets, it was performed a case study survey with computer science specialists that work on mobile programming. Only functional aspects were evaluated by a survey involving these specialists. In three of 9 questions Villa's \textit{Colors} has higher scores, but in 6 of them \textit{Memory Stroop} takes advantage, namely, in gamification proposal. That shows an evolution on gamification and other functional and non-functional requirements on the second version of the project.
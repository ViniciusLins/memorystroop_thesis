\chapter{Introduction}
\label{ch:introduction}

Attention-Deficit/Hyperactivity Disorder (ADHD) is a neurodevelopmental disturbance, including genetic and social factors, among others. According to Russell A. Barkley \citep{ADHDDay}, Clinical Professor of Psychiatry and Pediatrics, it was first described in 1797, and manifests itself with agitation, impassiveness, impulsive bodily control and concentration faults. There are even many others psychological problems main symptoms. That health disorder affects children and adults, but in different ways. Younger patients, generally, challenge difficulties in education and initial socialization process more than adult ones, besides these may suffer with serious relationship problems in work and family management. It afflicts between 5 and 7\% of world children, and 4 to 5\% adults \citep{ADHDDay}.

Some affected children may grow and preserve the same behaviour, cognitive and emotional complications all over their life. In this graduate thesis, the analysis will be focused only on young patients, particularly, before 12 years. Although many life scenarios may ridden by this mental ailment, effects of it may be detected mainly towards education, not only in academic grades, but also in everyday school life, including comportment in class and interpersonal relations with other classmates. Today ADHD children become a serious problem for educational systems, because their number have increased rapidly to higher figures. Their academic grades and interpersonal relationships are extensively affected by its symptoms, namely, focus difficult. In present informational society, in which flux of fast information turns life by itself agitated, that disorder may aggravate ADHD young patients' problems.  

Parents, teachers and pedagogues have a huge challenge to promote health conditions for the students in that situation. Common treatment of this disturbance involves strong controlled remedies. However, there are many controversies in using this medicaments. Like all medicaments, they have side effects. In order to seek alternatives, researchers must use creativity. One of these non-drug treatments is the use of brain games sessions.  But, before discussing that, it necessary to approach the use of games in ADHD diagnosis. According to Center of Disease Control and Prevention \citep{DSM-5}, there are some health signals and symptoms, grouped in DSM-5 check-list, that identifies presence of disturb among patients. 


  
\section{Motivation}
\label{sc:motivation}

Alternatives to strict medicament therapy includes gamification for health proposes. The word ``gamification'' means funny activities prototyped in video games' interface to serious aims, for example, for a human resources' sector of an enterprise. Here, it will be presented a incremented fork of a gamified application designed in this work to improve ADHD health conditions for children and teenagers. Many gamified applications could be suggested in this sense. For instance, memory, which is associated with various ADHD malfunctions impacts on afflicted young people, may be trained. Then, the gamification proposed in this present graduate thesis is directed to train memory activities by simple tasks, presets itself as promising initiative because searching on main popular playing records such as Google Play Store.

Many gamification softwares are projected to personal computers, but nowadays most used computational devices are mobile ones - namely smartphones and tablets. Developing mobile applications for hyperactive young students may collaborate in them academic performance, among others benefits. In Brazil, for example, from 2012 to 2013, number of mobile devices sold have increase 157\%, and they have been more sold than personal computers and conventional notebooks \citep{Villa}. Costumers' preferences for ``smaller computers'' could be inferred taking on account that they are, usually, cheaper and easier to carry and to use.  

% %\citep{idcCelulares, idcEstudoTablets}
Most of them are based in Android operating systems. Programmers involved in health informatics should be aware to produce more solutions for that platform. They, working conjunctively with psycho-pedagogues, may access easier young students, and offer them healthier life. On the other hand, near 1.9 million of children afflicted by ADHD in this country are waiting to use funny methods of treatment. As ABDA have stressed, it is commonest psychological illness among this strait of age  \citep{Villa}. Its necessary to address new computational efforts so that these millions of children around the world could play for benefit to their own health.


\section{Problem Statement}
\label{sc:problem} 

Many psychologists are searching others ways than remedies for controlling problems related to hyperactivity. This does not mean that they are totally disagree with prescription. The point lies in the almost virtual exclusivity of drug resources used among various techniques against the problem. There are considerable number of studies approaching such question.

One of useful alternatives is applying computerized games for ill children. Some authors have stressed relationship of the use of gamified software for diagnosis and treatment. It is a total non-invasive method to identifies, and sometime, to treat symptoms related to psychological disorders. There are games of famous corporation like Nintendo, with its NDSBA game among institutions developing brain games directed to hyperactive young people. Unfortunately, so little number of programmers are engaged in this type of medical computing initiative. For that reason, all initiative in this may must be seek anterior work which build new implementations. 

\begin{quote}
However, based on the literature, it seemed feasible that playing brain games such as NDSBA could stimulate the prefrontal cortex of students with ADHD, simulating the effects of stimulant medication, thus helping these students improve their ability to engage in classroom activities and perform tasks of executive function \citep{brainGames}. 
\end{quote}

Letting aside, by now, physiological details, the quotation above is useful to show the importance of gamification in treatment of ADHD. S\'{e}rgio Villa \citeyearpar{Villa}, a Brazilian computer scientist, developed an educational game for android devices and presented it in an graduate thesis - both remain paradigmatic to present research.

The game is called in Portuguese "Cores Beta" - Colors Beta -, which applies Stroop's test principles in order to exercise mental capacities of ADHD children, like low tolerance to frustration. 

As he could not develop statistical research with children using his game by himself, its necessary to develop and present the computational facilities involving them to check its validity. Different to ordinary games are not projected to this aspect and majority of them may not offer good results for promoting health to the referred children. 

With open source licence conditions, his application, is expanded here in his application and related research in order to investigate validity of this type of treatment. This graduate thesis aims to discuss and to present results obtained in quantitative research involving children using a fork of "Cores" for evaluate its psychotherapy possibilities. 


\section{Objectives}
\label{sc:objectives}

The objective of this work is to improve, with new software functionalities and search quantitative evidences, of efficiency, a gamified application in diagnosis and treatment towards ADHD children. 

Secondary objectives are: discuss the place of proposed application in context of gamified health alternatives to drug treatment; develop a gamification project targeting memory impacts of ADHD with Stroop Effect paradigm; and present reflections of possibilities of this application to ADHD diagnosis and treatment.

\section{Out of Scope}
\label{sc:outofscope}
This research, having so scarce resources beyond author's personal budget and initiative, could not map cerebral activity during experiments, neither follow up participants some moths after core sessions using medical application. Other experiments could involve parents and other agents in its tabulation. However, here it is foccused only in children that afflicted by the said disturbance.


\section{Statement of the Contributions}

This work contributes to Psychological/Psychiatric diagnosis and therapy, Social Computer Science and Pedagogics in the measures in which it offers them a gamified solution to identify and treatment. The gamification proposed involves a application that trains memory in a conflicting-stimuli condition (classical Stroop Effect): the health game Memory Stroop. Besides, theoretical contribution of this graduate thesis lies in offering computational and academic tools for new researches in non-drug approaches to ADHD-driven problems. Finally, minor contributions are set along this text, namely towards educational game modeling, gamification in health projects and interdisciplinary approaches in Information Systems projects.

\section{Graduate Thesis Structure}
\label{sc:structure}

In the Chapter \ref{ch:ADHD}, ``ADHD'', treats  psycho-pedagogical features of this mental condition. The Chapter \ref{ch:gamification} discuss gamification, medical video games and related questions. After that, in Chapter \ref{ch:development} it will be presented the application  and methodological principles adopted in codification. Finally Chapter \ref{ch:Experiment}  is the practical experiment of software validation.